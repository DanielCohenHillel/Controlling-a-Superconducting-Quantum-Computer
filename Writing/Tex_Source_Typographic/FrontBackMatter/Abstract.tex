% Abstract

%\renewcommand{\abstractname}{Abstract} % Uncomment to change the name of the abstract

\pdfbookmark[1]{Abstract}{Abstract} % Bookmark name visible in a PDF viewer

\begingroup
\let\clearpage\relax
\let\cleardoublepage\relax
\let\cleardoublepage\relax

\chapter*{Abstract}
  Quantum computing presents many challenges. Here, we try to tackle one of them: How do we control the quantum computer? What do you need to send to make what you want happen? How do you physically construct the control pulses and what do the pulses even look like? These are all question we are going to try to answer throughout this project.\newline\newline
   This manuscript starts with an introduction to quantum computing. Concepts such as the qubit and operations are introduced there.\newline\newline
   In the second chapter we introduce the quantum system. We use quantum optics and the Jaynes-Cumming model to model the behavior of the system.\newline\newline
   The third chapter is the main subject of this project, quantum optimal control and the \textbf{GR}adient \textbf{A}scent \textbf{P}ulse \textbf{E}ngineering (\textbf{GRAPE}) algorithm. In this chapter we show how can you use the system characterization we made in chapter 2 to find the pulses you need to send to control the quantum computer as you want.\newline\newline
   The fourth and last chapter describes the physical implementation of the transmission of the pulses we found in the previous chapter. We show how the entire system is connected, from the AWG\footnote{Arbitrary Waveform Generator} to the actual qubits. We discuss the challenges of creating the pulses and how to solve these problems.\newline\newline
   All the codes used in and created for this project, results, and references are available at\newline
\begin{center}
\url{https://github.com/DanielCohenHillel/Controlling-a-Superconducting-Quantum-Computer}
\end{center}

\endgroup			

\vfill
