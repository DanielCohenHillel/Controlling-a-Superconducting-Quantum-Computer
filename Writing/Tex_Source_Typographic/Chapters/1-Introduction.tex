\chapter{Introduction}
\section{What is a Quantum Computer?}
 A classical computer is, essentially, a calculator, not of real numbers but of \textit{binary numbers}. A \textit{binary digit} ("\textit{bit}" from now on) can be in one of two states, usually represented by 0 and 1. We can use \textit{logic gates} to control and manipulate bits to do various calculations. These are the building blocks of the classical computer. With the ability to do calculations on bits, and the ability to store bits in the memory we can construct a computer.
 
So what is a quantum computer then? Well, if the classical computer uses bits to do calculations, a quantum computer uses \textit{quantum bits} ("\textit{qubits}" from now on) for calculations. A qubit, much the same as a bit, has two states, a 0 state and a 1 state (denoted by $\ket{0}$ and $\ket{1}$ for reasons we'll see later). The principal difference between bits and qubits is that a qubit can be in a \textit{superposition} of the two states. We can use this property to our advantage by manipulating the state of a quantum computer so that desired outcomes interfere constructively, whereas undesired outcomes interfere destructively.

\section{Algorithms and Further motivation}
 \begin{quotation}
  \say{Nature isn't classical, dammit, and if you want to make a simulation of nature, you'd better make it quantum mechanical, and by golly it's a wonderful problem, because it doesn't look so easy.}
 \end{quotation}
\centerline{- Richard Feynman}
The possibilities that quantum computation allow are unprecedented. From simulation of drugs for developments of new cures to unbreakable encryption, quantum computing promises a lot. 

One of the most famous algorithms in quantum computing is \textit{Shor's algorithm}, a quantum algorithm for factoring large numbers. In classical computing the way to factor a number is to verify if small numbers divide it, one by one. Modern encryption method require you to factor a large number\footnote{This is called RSA encryption}. With classical computers this is a nearly impossible problem, since solving this problem requires a time that scales exponentially with the number of bits. This means that if we increase the size of the number that we need to factor by just just one bit, the time required to factorize the number increases by a factor of two! Shor's algorithm, on the other hand, uses the power of quantum computing to solve this problem in polynomial time!\footnote{The actual complexity is more detailed then this but it is meant to show the rough idea}. This means that increasing the size of a number to be factorized could quickly become an impossible task for a classical computer, it won't affect a quantum computer as much. 

While Shor's algorithm is a great example of the power of quantum computers, and it is also probably the most famous quantum algorithm, it is by no means the most interesting example. Decrypting messages and breaking the world's cryptography isn't really a good motivator to try to create quantum computers. Let's look at a more useful, optimistic algorithm, \textit{Grover's Algorithm}.

Grover's algorithm is a quantum algorithm that finds the unique input to a black box function that produces a particular output value, using just $o(\sqrt {N})$ evaluations of the function, where $N$ is the size of the function's domain. For a classical algorithm to do this it would take $o(N)$ evaluations. Therefore, Grover's algorithm provides us with a quadratic speedup. Roughly speaking, given function $y = f(x)$ Grover's algorithm finds $x$ when given a specific $y$. This algorithm could be used to search in databases quadratically faster then with a classical computer.

There are more quantum algorithms that were developed in the last several decades, and even more algorithms that have yet to be developed that might have impactful applications in the future.

\section{Qubits and Quantum Gates}
Physically, a qubit is a two level quantum system. We call the first level $\ket{0}$ and the second level $\ket{1}$. As we know from quantum mechanics, the qubit could be in a superposition of the two states.

Mathematically, we think of qubits as 2-dimensional vectors, where the first term corresponds to the $\ket{0}$ state and the second term corresponds to the $\ket{1}$ state, so a qubit in a state $\frac{1}{\sqrt{2}} \ket{0} + \frac{1}{\sqrt{2}} \ket{1}$ can be represented as
\[
\begin{pmatrix}
    \frac{1}{\sqrt{2}} \\
    \frac{1}{\sqrt{2}} 
\end{pmatrix} = \frac{1}{\sqrt{2}} \ket{0} + \frac{1}{\sqrt{2}} \ket{1}
\]
The complex pre-factors of each state are called probability amplitudes, since they are related to the probability of the qubit to be in that state. The probability is given by the absolute value of that state squared
\[
    P(\ket{i}) = \abs{\braket{i}{\psi}}^2
\]
In the example I just gave, the qubit has a $50\%$ chance to be in the $\ket{0}$ state and a $50\%$ chance to be in the $\ket{1}$ state.

In this world of qubits as vectors, we think of logic gates (\textit{quantum gates}), as unitary matrices. When the qubit goes through a logic gate, the resulting state is obtained by multiplying the initial state by the matrix. Let's look at an example for one of the simplest logic gates we have in classical computing, the NOT gate. A quantum implementation of the NOT gate takes $\ket{0}$ to $\ket{1}$ and $\ket{1}$ to $\ket{0}$). The matrix that achieves this is 
\[
\begin{pmatrix}
    0 & 1 \\
    1 & 0
\end{pmatrix}
\]
Known as $\hat{\sigma}_x$ (Pauli matrix $X$). As a simple example to see how this works, if we input $\ket{0}$ into the NOT quantum gate, we get as a result
\[
NOT \ket{0} = 
\begin{pmatrix}
    0 & 1 \\
    1 & 0
\end{pmatrix}
\begin{pmatrix}
    1 \\
    0
\end{pmatrix} = 
\begin{pmatrix}
    0 \\
    1
\end{pmatrix} = \ket{1}
\]
As we expected, NOT $\ket{0}$ is $\ket{1}$. There are infinite 1-qubit quantum gates, while there are only four possible one bit gates on a classical computer\footnote{These are: Identity, NOT, always 1, and always 0}.

The last thing we need to know to understand the basic of quantum computing, is how to represent multiple qubits. If we have several of qubits in our system, we think of all the qubits together as one vector that is the \textit{tensor product}\footnote{Represented as a Kronecker product for the dimensions sake} of all the qubits. Let's say we have a $\ket{0}$ and $\ket{1}$ qubits in our system, we represent that by $\ket{01}$ and it is equal to
\[\ket{01} = \ket{0} \otimes \ket{1} =
\begin{pmatrix}
    0 \\
    1 \\
    0 \\
    0
\end{pmatrix}\]
% A quantum gate on multiple qubits is simply a larger matrix.

The tensor product of N qubits has $2^N$ coefficients! This is yet another clue of the power that quantum computers have compared to classical computers. A quantum gate on multiple qubits is thus a  $2^N$ by $2^N$ square matrix.

Now that we have the basic tools of quantum computing, we can use them to get motivation for the amazing things quantum computers can do

\section{Superconducting Quantum Computers}
The physical implementation of the qubit itself isn't the subject of this project but we still look on how would one implement such a device. A problem we have to face when making a quantum computer is what physical phenomenon would be the qubit. We need some sort of two level system that we can easily measure and manipulate, while also staying coherent\footnote{Coherence is a big subject, you can think of it as a fancy way of saying that the qubit still holds information without being corrupted. This is the main problem facing quantum computing right now} and usable. For classical computers we already have this figured out for years, a bit is a voltage on a wire, 1 is when there is voltage on the wire and 0 if there's none, simple. For a quantum computer this is much more complicated, there are many quantum phenomena we can use as our qubit, such as the energy level of an atom, the spin of an electron, the polarization of photons and so on. It is not so obvious what should be the physical realisation of the qubit. This project is about a \textit{superconducting} quantum computer, with superconducting  qubits.

Superconducting qubits are microwave circuits in which the cooper-pair condensate effectively behaves as a one-dimensional quantized particle. By inserting Josephson junctions, the circuit can be made nonlinear, allowing us to isolate the qubit from higher-energy levels and treat it as a two level system (instead of the many level system that it really is).

This topic is covered in appendix \ref{appen:LC}, refer there for any additional information, preferably read the appendix after reading chapter \ref{chap:quantum-optics}.

\section{References and Further Readings}
The main references for this chapter were two great books on quantum information I recommend everyone to read. The first is the well known, \textit{de-facto} book on quantum computation and information. "\textbf{Quantum Computation and Quantum Information}" written by Michael Nielsen and Isaac Chuang also known as "Mike and Ike". This book covers everything.

The other book I used was "\textbf{Quantum Computing for Computer Scientists}" by Mirco A. Mannucci and Noson S. Yanofsky. It has, in my opinion, clearer explanations on the pure mathematical nature of quantum information, although it is not as comprehensive as "Mike and Ike".