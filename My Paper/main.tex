

\documentclass{article}
\usepackage[utf8]{inputenc}
\usepackage{physics}

\usepackage{color}   %May be necessary if you want to color links
\usepackage{hyperref}
\hypersetup{
    colorlinks=true, %set true if you want colored links
    linktoc=all,     %set to all if you want both sections and subsections linked
    linkcolor=black,  %choose some color if you want links to stand out
}

\title{Controlling a Superconducting Quantum Computer}
\author{Daniel Cohen Hillel}
\date{}

\begin{document}

\maketitle

\newpage

\tableofcontents

\newpage

\section{Introduction}

\subsection{What is a Quantum Computer?}
We'll assume the reader understands the basics of computing and quantum mechanics, here's a brief overview. \newline
 A classical computer is, essentially, a calculator, not of "Regular" numbers but of \textit{binary numbers} \footnote{ add further reading about binary numbers}. A \textit{binary digit}("\textit{bits}" from now on) can be in one of two states, usually represented by 0 and 1. We can use \textit{logic gates} to control and manipulate bits to do all kinds of calculations\footnote{Additional information about bit calculation}. This is the building blocks of the classical computer, with the ability to do calculation with bits, and the ability to store bits in the memory we are able to construct a computer.\newline \par
So what is a quantum computer then? Well, if the classical computer uses bits to do calculations, a quantum computer uses \textit{quantum bits}("\textit{qubits}" from now on) for it's calculations. A qubit, much the same as a bit, has 2 states, a 0 state and a 1 state(notated $\ket{0}$ and $\ket{1}$ for reasons we'll see later), the difference is that a qubit can be in a \textit{superposition} of the 2 states, so the qubit has essentially and infinite amount of possiable states

\subsection{Qubits and Quantum Gates}

\subsection{Algorithms and Further motivation}

\subsection{Superconducting Quantum Computers}

\newpage

\end{document}
